 \chapter{\centering Abstract}
 In this work we present a theoretical study of some magnetic properties of four two-dimensional materials: PtSe\textsubscript{2}, PtS\textsubscript{2}, VSe\textsubscript{2} and VS\textsubscript{2} by means of first-principles techniques. Platinum-based materials (PtSe\textsubscript{2} and PtS\textsubscript{2}) are nominaly non-magnetic materials but exhibit  very large spin-orbit coupling ($\approx 230~meV$) whereas Vanadium-based materials (VSe\textsubscript{2} and VSe\textsubscript{2}) are ferromagnetic. Subsequently, the effects of the application of a mechanical deformation and the inclusion of a vacancy (Pt or V) on the magnetic activity of these materials are also studied.
\newline
 \par On the experimental side, the most employed experimental technique to study the magnetic properties of materials is the magneto-optical Kerr effect, which is correlated with the changes in the light reflected by the magnetic activity of a material and which is mainly reflected with the changes in the polarization and intensity of the light. This phenomenon is mainly used to measure the hysteresis of the ferromagnetic material and from which the magnetic anisotropy and the cohersive fields are retrieved. Furthermore, if the Kerr effect is measured as a function of the wavelength of the incident light beam, it provides valuable information about the effects of the magnetic field on the electronic structure of the material. 
\newline
 \par Given the theoretical and experimental motivations mentioned above, this work also presents the implementation of a magneto-optical Kerr effect spectroscopy system in longitudinal configuration and is applied  to a CoFeB film deposited on a SiO\textsubscript{2} substrate.

