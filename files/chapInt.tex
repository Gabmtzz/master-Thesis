\chapter*{Introducci\'on}
\addcontentsline{toc}{chapter}{Introducci\'on}
Los procesos físicos basados en el spin representan una parte fundamental en el estudio de la materia condensada y aunque existen todav\'ia algunas cuestiones que no han sido explicadas satisfactoriamente, como el acople spin-\'orbita, la interacci\'on fot\'on-spin, el ordenamiento de spines en bajas dimensiones \cite{spin-2016}, entre otros, se ha planteado la spintr\'onica como el estudio interdisciplinario de dispositivos que funcionan en base a estos fen\'omenos y que puedan ser utilizados en dispositivos para el guardado y procesado digital de la informaci\'on as\'i como en el desarrollo de nuevos dispositivos que no generan exceso de calor  debido a que no se basan en un transporte de carga \cite{spintronics-2016}.
\newline
\par Es importante mencionar a las excitaciones  colectivas de los spines llamadas ondas de spin \cite{spinW-2009}, cuya cuantizaci\'on se le conoce como magn\'on \cite{magnon-2020}. Dichas ondas se pueden propagar en  un material anti- o ferromagn\'etico y esencialmente se deben a la precesi\'on  del momento magn\'etico cuya frecuencia se encuentra en el rango de las microondas  y por este motivo pueden ser utilizados en el procesamiento de se\~nales \cite{magnon-2020}.
\newline
\par Desde el descubrimiento del grafeno en el 2004 \cite{grap-2005} han aparecido distintos materiales bidimensionales con propiedades interesantes dado al confinamiento de su estructura de bandas, por lo cual es de gran inter\'es poder desarrollar dispositivos spintr\'onicos o magn\'onicos en base a \'estos \cite{spin-2016}. Una familia de estos materiales son los denominados dicalcogenuros de metales de transici\'on  (TMDs, por sus sigas en ingl\'es) \cite{tmds-2013}, los cuales se basan en la f\'ormula MX\textsubscript{2}, en donde M representa al metal de transici\'on y X al calc\'ogeno. Estos materiales tienen una estructura hexagonal  u octaedral y su celda unitaria se puede considerar como una tricapa formada por una de metales de transici\'on colocada entre dos capas de calc\'ogenos \cite{tmds-2013}. Dichos materiales tienen como principal caracter\'istica que algunos presentan una brecha prohibida cuando se tiene una monocapa; adem\'as  exhibir un fuerte  efecto de acople spin-\'orbita, por lo cual los hace muy atractivos para aplicaciones para spintr\'onica y magn\'onica \cite{tmds-2016, spin-2016}. Por lo normal estos materiales son no magn\'eticos, aunque es posible inducir un momento magn\'etico por medio de  defectos \cite{tmds-2016}.
\newline 
\par Una de las t\'ecnicas experimentales mas utilizadas para estudiar las propiedades magn\'eticas de los materiales es el efecto Kerr magneto-\'optico, el cual est\'a relacionado con los cambios en la luz reflejada por un material magn\'etico y que se refleja principalmente con los cambios en la polarizaci\'on e intensidad de la luz \cite{MOp-1997}. Este fenómeno es utilizado para obtener la hist\'eresis del material ferromagn\'etico y de la cual se puede medir la anisotrop\'ia magn\'etica y los campos de cohersi\'on, principalmente \cite{MOp-2008}. M\'as a\'un, si el efecto Kerr se mide en funci\'on de la longitud de onda del haz incidente, \'este provee informaci\'on valiosa acerca de los efectos del campo  magn\'etico en la estructura electr\'onica del material \cite{MOp-1997}.
\newline
\par En el presente trabajo se estudiar\'an las propiedades magn\'eticas de cuatro materiales bidimensionales: PtSe\textsubscript{2}, PtS\textsubscript{2}, VSe\textsubscript{2} y VS\textsubscript{2} por medio de t\'ecnicas de primeros principios. Los materiales basados con Platino (PtSe\textsubscript{2} y PtS\textsubscript{2}) son materiales no magn\'eticos pero que tienen un acople spin-\'orbita muy grande ($~230~meV$) \cite{tmds-2016, PhysRevB.103.125409} y los materiales basados en Vanadio  (VSe\textsubscript{2} y VS\textsubscript{2}) son ferromagn\'eticos \cite{tmds-2016}. Posteriormente, se estudia el efecto de una deformación mec\'anica en la magnetizaci\'on de estos materiales. Finalmente se trabaj\'o en la implementaci\'on de un sistema de espectroscop\'ia de efecto Kerr magneto-\'optico en configuraci\'on longitudinal y se aplica en un apel\'icula de CoFeB crecida en un substrato de SiO\textsubscript{2}.
\newline
\par La organizaci\'on de este trabajo consiste en el cap\'itulo \ref{cap:DFT}, en donde se presenta en detalle los fundamentos de la teor\'ia funcional de la densidad, la cual es utilizada para calcular las propiedades magn\'eticas de los materiales mencionados anteriormente. En el cap\'itulo \ref{cap:Kerr} se explica el efecto Kerr magneto-\'optico, en el cap\'itulo \ref{cap:Metodos} se muestran los m\'etodos computacionales y experimentales utilizados en la presente tesis. Finalmente, en los cap\'itulos \ref{cap:Sim} y \ref{cap:exp} se muestran los resultados del estudio te\'orico y de las mediciones de efecto Kerr en CoFeB, respectivamente. 
