\section{conclusiones}
\frame{
	\frametitle{Conclusiones}
	\begin{itemize}
		\item Se comenz\'o con el estudio de materiales bidimensionales llamados metales de dicalcogenuros de transición basados en  Platino y  Vanadio y utilizando el Azufre y Selenio como calc\'ogenos. 
		\item Al momento de estudiar estos materiales sin defectos, se observ\'o en el PtSe\textsubscript{2} y PtS\textsubscript{2} que no existe una magnetizaci\'on neta a pesar de que se observa un efecto spin-\'orbita considerable asociado principalmente al \'atomo  Platino. En el caso contrario,  para VSe\textsubscript{2} y  VS\textsubscript{2} si existe una magnetizaci\'on distinta de cero. Observando el comportamiento de la magnetizaci\'on en estos materiales se pudo notar que el cambio de Selenio o Azufre no afecta considerablemente a las propiedades magn\'eticas y por lo tanto  el cambio recae en el metal de transici\'on (Pt y V).
		
	\end{itemize}
}
\frame{
	\begin{itemize}
		\item se estudi\'o el efecto de introducir una vacancia del metal de transici\'on y se observ\'o que en el caso de los materiales con Platino, aparece una magnetizaci\'on que ronda en el valor de $2.5 \mu_{B}/celda$
		\item Para  el caso de los materiales con vanadio se  pudo observar que la magnetizaci\'on se reduce considerablemente a pesar de que la deformaci\'on del sistema es muy peque\~na.
		\item se pudo investigar el efecto que tiene una deformaci\'on mec\'anica isotr\'opica y anisotr\'opica en VSe\textsubscript{2} y VS\textsubscript{2} sin defectos; as\'i como en  PtSe\textsubscript{2} y PtS\textsubscript{2} con una vacancia de Platino.
	\end{itemize}
}
\frame{
		\frametitle{Conclusiones}
	\begin{itemize}
		\item Se logr\'o desarrollar un sistema de espectroscopia de Efecto Kerr magneto-\'optico partiendo de un montaje ya dise\~nado anteriormente y se desarroll\'o la instrumentaci\'on virtual para su control. 
		\item  fue necesario medir elipsometr\'ia a 70 \degree a dicho beamsplitter y   se escribi\'o un c\'odigo en Python para poder adquirir la rotaci\'on ($\theta_k$) y la elipticidad ($\eta_k$)  Kerr tanto de las mediciones de hist\'eresis como del espectro a incidencia casi normal con el beamsplitter a 45 \degree.
		\item se midi\'o la hist\'eresis y espectro Kerr de una muestra de CoFeB depositada sobre Sio\textsubscript{2}
	\end{itemize}
}