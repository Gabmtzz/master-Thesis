\chapter{Conclusiones} \label{Conc:cap}
El interés actual por investigar algunas de las propiedades magnéticas en diversos materiales sintetizados en película delgada, ha motivado su exploración en nuevas propuestas; en particular, en espintr\'onica con dirección a funcionalidades basadas en efectos magn\'onicos, que se forman a partir de excitaciones en sistemas ferromagnéticos.
Para tal fin, en este trabajo se propuso una metodología que consta en la investigación fundamental \textit{ab initio}, usando Quantum-Esspresso y aparte la implementación y desarrollo de un sistema para la medici\'on de efecto Kerr. El trabajo se puede resumir como sigue:

Se comenz\'o con el estudio de materiales bidimensionales llamados metales de dicalcogenuros de transición basados en  Platino y  Vanadio y utilizando el Azufre y Selenio como calc\'ogenos. Al momento de estudiar estos materiales sin defectos, se observ\'o en el PtSe\textsubscript{2} y PtS\textsubscript{2} que no existe una magnetizaci\'on neta a pesar de que se observa un efecto spin-\'orbita considerable asociado principalmente al \'atomo  Platino. En el caso contrario,  para VSe\textsubscript{2} y  VS\textsubscript{2} si existe una magnetizaci\'on distinta de cero. Observando el comportamiento de la magnetizaci\'on en estos materiales se pudo notar que el cambio de Selenio o Azufre no afecta considerablemente a las propiedades magn\'eticas y por lo tanto  el cambio recae en el metal de transici\'on (Pt y V). 
%\newline
%\par 
 Posteriormente se estudi\'o el efecto de introducir una vacancia del metal de transici\'on y se observ\'o que en el caso de los materiales con Platino, aparece una magnetizaci\'on que ronda en el valor de $2.5 \mu_{B}/celda$; adem\'as fue posible notar que la distancia entre los \'atomos de Selenio o Azufre y Platino se reducen aunque no existe un cambio considerable en la estructura de este material.  Para  el caso de los materiales con vanadio se  pudo observar que la magnetizaci\'on se reduce considerablemente a pesar de que la deformaci\'on del sistema es muy peque\~na.
%\newline
%\par
 Finalmente, se pudo investigar el efecto que tiene una deformaci\'on mec\'anica isotr\'opica y anisotr\'opica en VSe\textsubscript{2} y VS\textsubscript{2} sin defectos; as\'i como en  PtSe\textsubscript{2} y PtS\textsubscript{2} con una vacancia de Platino. 
 \newline
 \par En relaci\'on a materiales con Vanadio se observa que bajo una deformaci\'on isotr\'opica var\'ia  la magnetizaci\'on de estos materiales considerablemente: para  el caso de la anisotr\'opica esta se mantiene constante. Dichos  dos fen\'omenos se pueden explicar si se considera el cambio en las distancias entre el \'atomo de vanadio y los \'atomos de Azufre o Selenio notándose   que \'estas no cambian considerablemente  en el caso anisotr\'opico. En el caso isotr\'opico estas  cambian considerablemente y muestran un comportamiento paralelo a la magnetizaci\'on, y entonces debido a que el cambio en dicha distancia, tambi\'en  cambia la intensidad de interacci\'on entre los enlaces qu\'imicos. para materiales con Platino se muestra que en general los cambios en la magnetizaci\'on son muy similares, ya que en un rango de la deformaci\'on $\varepsilon$, para el caso isotr\'opico, y $\varepsilon_x$, para el anisotr\'opico, se mantiene la magnetizaci\'on casi constante.  En ambos casos , para cierto valor de la deformaci\'on, \'esta se reduce a cero coincidiendo con el mayor acercamiento entre los \'atomos calc\'ogenos, lo que provocar\'ia la creaci\'on de nuevos enlaces entre \'estos.
 \newline
 \par Finalmente se logr\'o desarrollar un sistema de espectroscopia de Efecto Kerr magneto-\'optico partiendo de un montaje ya dise\~nado anteriormente y se desarroll\'o la instrumentaci\'on virtual para su control. Para asegurarnos  del correcto funcionamiento de este sistema se realizaron mediciones de hist\'eresis en una muestra de CoFeB y una vez que fue posible obtenerlas, se procedi\'o a adquirir los valores de rotaci\'on y elipticidad en funci\'on de la longitud de onda. A partir  del an\'alisis de Jones fue posible notar que el separador de haz (beamsplitter) alteraba las mediciones y para poder realizar las correcciones necesarias fue necesario medir elipsometr\'ia a 70 \degree a dicho beamsplitter y   se escribi\'o un c\'odigo en Python para poder adquirir la rotaci\'on ($\theta_k$) y la elipticidad ($\eta_k$)  Kerr tanto de las mediciones de hist\'eresis como del espectro a incidencia casi normal con el beamsplitter a 45 \degree.
 \newline
 \par Como trabajo futuro se propone tratar de estudiar la espectroscopia de efecto Kerr en materiales bidimensionales utilizando t\'ecnicas mas avanzadas que DFT, tales como la teor\'ia funcional de la densidad dependiente del tiempo (TDDFT)\cite{PhysRevLett.52.997}, aproximaci\'on GW\cite{Aryasetiawan_1998,PhysRevB.64.235106} y las ecuaciones de Bethe-Salpeter\cite{PhysRev.84.1232} para estudiar las propiedades \'opticas de los materiales; adem\'as se desea calcular la dispersión de magnones en estos materiales utilizando las teor\'ias antes mencionadas. En cuanto al sistema de espectroscopia se desea completar la automatizaci\'on del sistema dise\~nando un sistema de control desde una computadora de las bobinas y de esta forma tener un control mas preciso del campo magn\'etico aplicado. Finalmente es necesario realizar un an\'alisis de un ,medio estratificado para obtener la respuesta Kerr magneto-\'optica  de la película delgada constituyente. Se espera que dicho trabajo estimule investigaci\'on y desarrollo de nuevas propuestas en materiales y dispositivos magn\'onicos. 