\chapter{\centering Resumen}
\par En este trabajo se presenta un estudio te\'orico de  propiedades magn\'eticas de cuatro materiales bidimensionales: PtSe\textsubscript{2}, PtS\textsubscript{2}, VSe\textsubscript{2} y VS\textsubscript{2} por medio de c\'alculos basados  en primeros principios. Los materiales compuestos con Platino (PtSe\textsubscript{2} y PtS\textsubscript{2}) son sistemas no magn\'eticos pero que tienen un acople spin-\'orbita considerable ($\approx 230~meV$) y los materiales basados en Vanadio  (VSe\textsubscript{2} y VS\textsubscript{2}) son ferromagn\'eticos. Posteriormente, adem\'as, se estudian el efectos que tienen, tanto  la aplicaci\'on de  una deformación mec\'anica y la inclusi\'on  de una vacancia (Pt o V), en la actividad magn\'etica  de estos materiales.
\newline
\par Por el lado experimental, una de las herramientas mas utilizadas para estudiar las propiedades magn\'eticas de los materiales es el efecto Kerr magneto-\'optico, el cual est\'a correlacionado con los cambios en la luz reflejada por la actividad magn\'etica de un material y que se refleja principalmente con los cambios en la polarizaci\'on e intensidad de la luz. Este fenómeno es utilizado principalmente para obtener la hist\'eresis del material ferromagn\'etico y de la cual se puede medir la anisotrop\'ia magn\'etica y los campos de cohersi\'on, principalmente. M\'as a\'un, si el efecto Kerr se mide en funci\'on de la longitud de onda del haz incidente, \'este provee informaci\'on valiosa acerca de los efectos del campo  magn\'etico en la estructura electr\'onica del material. 
\newline
\par Dado la motivaci\'on te\'orica y experimental mencionadas, en este trabajo  se presenta adem\'as  la implementaci\'on de un sistema de espectroscopia de efecto Kerr magneto-\'optico en configuraci\'on longitudinal y se aplica en una pel\'icula de CoFeB depositada en un substrato de SiO\textsubscript{2}.