\documentclass[12pt,a4paper]{book}
\usepackage{fancyhdr}

\pagestyle{fancy}
\renewcommand{\chaptermark}[1]{\markboth{\thechapter.#1}{}}
\renewcommand{\sectionmark}[1]{\markright{\thesection.#1}}
\fancyhead[le,ro]{\textbf{\thepage}}
\fancyhead[re]{\textbf{\leftmark}}
\fancyhead[lo]{\textbf{\rightmark}}
%\fancyfoot[le]{\textbf{materiales bidimensionales}
%\fancyfoot[ro]{\textbf{}}
\fancyfoot[ro]{}
\renewcommand{\headrulewidth}{0.15mm}
\renewcommand{\footrulewidth}{0.15mm}
\addtolength{\headwidth}{\marginparsep}
\addtolength{\headwidth}{\marginparwidth}
\usepackage[spanish]{babel}
\begin{document}
	\chapter*{Declaraci\'on de autor\'ia}
	\pagestyle{empty}
	yo declaro que hice este trabajo con ar
	\chapter*{Resumen}
	\pagestyle{empty}
	yo declaro que hice este trabajo con ar
	\chapter*{Abstract}
	\pagestyle{empty}
	yo declaro que hice este trabajo con ar
	\chapter*{Agradecimientos}
	\pagestyle{empty}
	
	\tableofcontents
	
	\chapter{Introducci\'on}
	en el presente trabajo se muestra un estudio  te\'orico de materiales bidimensionales para poder encontrar su respuesta 
	
	\chapter{Teor\'ia funcional de la densidad}
	
	\section{teoremas de Hohenberg-Kohn}
	
	\section{ecuaciones de Kohn-Sham}
	\section{Funcional de intercambio y correlaci\'on (XC)}
	\subsection{propiedades del funcional XC}
	\subsection{correcci\'on de gradientes generalizados}
	\subsubsection{aproximaci\'on PBE}
	\section{pseudopotenciales}
	\section{correcciones relativista escalar y completamente relativista}
	\section{formalismo espacio-k y soluci\'on del sistema de ecuaciones de Kohn-Sham}
	\section{cálculo de fuerza por medio del teorema de Hellman-Feyman}
	\subsection{algoritmo de Broyden-Fletcher-Goldfarb-Shanno (BFGS)}
	
	\chapter{Materiales bidimensionales formados por dicalcogenuros de metales de trancisi\'on}
	\section{propiedades geom\'etricas}
	\section{propiedades electr\'onicas}
	\section{propiedades magn\'eticas}
	
	\chapter{M\'etodos}
	\section{m\'etodos para simulaciones }
	\subsection{elaboraci\'on de superceldas}
	\subsection{descripción del software Quantum Espresso}
	\subsubsection{detalles computacionales}
	\section{Sistema de espectroscop\'ia por efecto Kerr magneto-\'optico}
	
	\chapter{Resultados  y an\'alisis de simulaciones }
	\section{estudio de propiedades magn\'eticas en mono-capas sin defectos }
	\subsection{estrutura electr\'onica}
	\subsection{propiedades magn\'eticas}
	\section{estudio de vacancia de metal de transici\'on en material bidimensional}
	\subsection{estructura electrónica y propiedades magn\'eticas}
	\subsection{efecto del strain en las propiedades magn\'eticas}
	
	\chapter{resultados mediciones de efecto Kerr}
	
	\chapter{conclusiones}
	
\end{document}